%
% 6.006 problem set 5
%
\documentclass[12pt,twoside]{article}

% Cross-references for handout numbers.

% Updated to include SMA course for Fall 2001 -- cel

\newcommand{\name}{}


\usepackage{latexsym}
%\usepackage{bbm}
\usepackage{times,url}
\usepackage{clrscode3e}

\newcommand{\mitst}[1]{\begin{description}
\item[MIT students:] #1
\end{description}}
\newcommand{\smast}[1]{\begin{description}
\item[SMA students:] #1
\end{description}}

\newcommand{\profs}{Professors Erik Demaine and Srini Devadas}
\newcommand{\subj}{6.006}

\newlength{\toppush}
\setlength{\toppush}{2\headheight}
\addtolength{\toppush}{\headsep}

\newcommand{\htitle}[2]{\noindent\vspace*{-\toppush}\newline\parbox{6.5in}
{\textit{Introduction to Algorithms: 6.006}\hfill\name\newline
Massachusetts Institute of Technology \hfill #2\newline
\profs\hfill #1 \vspace*{-.5ex}\newline
\mbox{}\hrulefill\mbox{}}\vspace*{1ex}\mbox{}\newline
\begin{center}{\Large\bf #1}\end{center}}

\newcommand{\handout}[2]{\thispagestyle{empty}
 \markboth{#1}{#1}
 \pagestyle{myheadings}\htitle{#1}{#2}}

\newcommand{\htitlewithouttitle}[2]{\noindent\vspace*{-\toppush}\newline\parbox{6.5in}
{\textit{Introduction to Algorithms}\hfill#2\newline
Massachusetts Institute of Technology \hfill 6.006\newline
%Singapore-MIT Alliance \hfill SMA5503\newline
\profs\hfill Handout #1\vspace*{-.5ex}\newline
\mbox{}\hrulefill\mbox{}}\vspace*{1ex}\mbox{}\newline}

\newcommand{\handoutwithouttitle}[2]{\thispagestyle{empty}
 \markboth{Handout \protect\ref{#1}}{Handout \protect\ref{#1}}
 \pagestyle{myheadings}\htitlewithouttitle{\protect\ref{#1}}{#2}}

\newcommand{\exam}[2]{% parameters: exam name, date
 \thispagestyle{empty}
 \markboth{\subj\ #1\hspace{1in}Name\hrulefill\ \ }%
          {\subj\ #1\hspace{1in}Name\hrulefill\ \ }
 \pagestyle{myheadings}\examtitle{#1}{#2}
 \renewcommand{\theproblem}{Problem \arabic{problemnum}}
}
\newcommand{\examsolutions}[3]{% parameters: handout, exam name, date
 \thispagestyle{empty}
 \markboth{Handout \protect\ref{#1}: #2}{Handout \protect\ref{#1}: #2}
% \pagestyle{myheadings}\htitle{\protect\ref{#1}}{#2}{#3}
 \pagestyle{myheadings}\examsolutionstitle{\protect\ref{#1}} {#2}{#3}
 \renewcommand{\theproblem}{Problem \arabic{problemnum}}
}
\newcommand{\examsolutionstitle}[3]{\noindent\vspace*{-\toppush}\newline\parbox{6.5in}
{\textit{Introduction to Algorithms}\hfill#3\newline
Massachusetts Institute of Technology \hfill 6.006\newline
%Singapore-MIT Alliance \hfill SMA5503\newline
\profs\hfill Handout #1\vspace*{-.5ex}\newline
\mbox{}\hrulefill\mbox{}}\vspace*{1ex}\mbox{}\newline
\begin{center}{\Large\bf #2}\end{center}}

\newcommand{\takehomeexam}[2]{% parameters: exam name, date
 \thispagestyle{empty}
 \markboth{\subj\ #1\hfill}{\subj\ #1\hfill}
 \pagestyle{myheadings}\examtitle{#1}{#2}
 \renewcommand{\theproblem}{Problem \arabic{problemnum}}
}

\makeatletter
\newcommand{\exambooklet}[2]{% parameters: exam name, date
 \thispagestyle{empty}
 \markboth{\subj\ #1}{\subj\ #1}
 \pagestyle{myheadings}\examtitle{#1}{#2}
 \renewcommand{\theproblem}{Problem \arabic{problemnum}}
 \renewcommand{\problem}{\newpage
 \item \let\@currentlabel=\theproblem
 \markboth{\subj\ #1, \theproblem}{\subj\ #1, \theproblem}}
}
\makeatother


\newcommand{\examtitle}[2]{\noindent\vspace*{-\toppush}\newline\parbox{6.5in}
{\textit{Introduction to Algorithms}\hfill#2\newline
Massachusetts Institute of Technology \hfill 6.006 Fall 2009\newline
%Singapore-MIT Alliance \hfill SMA5503\newline
\profs\hfill #1\vspace*{-.5ex}\newline
\mbox{}\hrulefill\mbox{}}\vspace*{1ex}\mbox{}\newline
\begin{center}{\Large\bf #1}\end{center}}

\newcommand{\grader}[1]{\hspace{1cm}\textsf{\textbf{#1}}\hspace{1cm}}

%\newcommand{\points}[1]{[#1 points]\ }
\newcount\pointcount
\newcommand{\points}[1]{\pointcount=#1\relax \ifnum\pointcount=1 [#1 point]\else [#1 points]\fi\ }

\newcommand{\parts}[1]
{
  \ifnum#1=1
  (1 part)
  \else
  (#1 parts)
  \fi
  \ 
}

\newcommand{\bparts}{\begin{problemparts}}
\newcommand{\eparts}{\end{problemparts}}
\newcommand{\ppart}{\problempart}

%\newcommand{\lg} {lg\ }

\setlength{\oddsidemargin}{0pt}
\setlength{\evensidemargin}{0pt}
\setlength{\textwidth}{6.5in}
\setlength{\topmargin}{0in}
\setlength{\textheight}{8.5in}


\renewcommand{\cases}[1]{\left\{ \begin{array}{ll}#1\end{array}\right.}
\newcommand{\cif}[1]{\mbox{if $#1$}}
\newcommand{\cwhen}[1]{\mbox{when $#1$}}

\newcounter{problemnum}
\newcommand{\theproblem}{Problem \theproblemsetnum-\arabic{problemnum}}
\newenvironment{problems}{
        \begin{list}{{\bf \theproblem. \hspace*{0.5em}}}
        {\setlength{\leftmargin}{0em}
         \setlength{\rightmargin}{0em}
         \setlength{\labelwidth}{0em}
         \setlength{\labelsep}{0em}
         \usecounter{problemnum}}}{\end{list}}
\makeatletter
\newcommand{\problem}[1][{}]{\item \let\@currentlabel=\theproblem \textbf{#1}}
\makeatother

\newcounter{problempartnum}[problemnum]
\newenvironment{problemparts}{
        \begin{list}{{\bf (\alph{problempartnum})}}
        {\setlength{\leftmargin}{2.5em}
         \setlength{\rightmargin}{2.5em}
         \setlength{\labelsep}{0.5em}}}{\end{list}}
\newcommand{\problempart}{\addtocounter{problempartnum}{1}\item}

\newenvironment{truefalseproblemparts}{
        \begin{list}{{\bf (\alph{problempartnum})\ \ \ T\ \ F\hfil}}
        {\setlength{\leftmargin}{4.5em}
         \setlength{\rightmargin}{2.5em}
         \setlength{\labelsep}{0.5em}
         \setlength{\labelwidth}{4.5em}}}{\end{list}}

\newcounter{exercisenum}
\newcommand{\theexercise}{Exercise \theproblemsetnum-\arabic{exercisenum}}
\newenvironment{exercises}{
        \begin{list}{{\bf \theexercise. \hspace*{0.5em}}}
        {\setlength{\leftmargin}{0em}
         \setlength{\rightmargin}{0em}
         \setlength{\labelwidth}{0em}
         \setlength{\labelsep}{0em}
        \usecounter{exercisenum}}}{\end{list}}
\makeatletter
\newcommand{\exercise}{\item \let\@currentlabel=\theexercise}
\makeatother

\newcounter{exercisepartnum}[exercisenum]
%\newcommand{\problem}[1]{\medskip\mbox{}\newline\noindent{\bf Problem #1.}\hspace*{1em}}
%\newcommand{\exercise}[1]{\medskip\mbox{}\newline\noindent{\bf Exercise #1.}\hspace*{1em}}

\newenvironment{exerciseparts}{
        \begin{list}{{\bf (\alph{exercisepartnum})}}
        {\setlength{\leftmargin}{2.5em}
         \setlength{\rightmargin}{2.5em}
         \setlength{\labelsep}{0.5em}}}{\end{list}}
\newcommand{\exercisepart}{\addtocounter{exercisepartnum}{1}\item}


% Macros to make captions print with small type and 'Figure xx' in bold.
\makeatletter
\def\fnum@figure{{\bf Figure \thefigure}}
\def\fnum@table{{\bf Table \thetable}}
\let\@mycaption\caption
%\long\def\@mycaption#1[#2]#3{\addcontentsline{\csname
%  ext@#1\endcsname}{#1}{\protect\numberline{\csname 
%  the#1\endcsname}{\ignorespaces #2}}\par
%  \begingroup
%    \@parboxrestore
%    \small
%    \@makecaption{\csname fnum@#1\endcsname}{\ignorespaces #3}\par
%  \endgroup}
%\def\mycaption{\refstepcounter\@captype \@dblarg{\@mycaption\@captype}}
%\makeatother
\let\mycaption\caption
%\newcommand{\figcaption}[1]{\mycaption[]{#1}}

\newcounter{totalcaptions}
\newcounter{totalart}

\newcommand{\figcaption}[1]{\addtocounter{totalcaptions}{1}\caption[]{#1}}

% \psfigures determines what to do for figures:
%       0 means just leave vertical space
%       1 means put a vertical rule and the figure name
%       2 means insert the PostScript version of the figure
%       3 means put the figure name flush left or right
\newcommand{\psfigures}{0}
\newcommand{\spacefigures}{\renewcommand{\psfigures}{0}}
\newcommand{\rulefigures}{\renewcommand{\psfigures}{1}}
\newcommand{\macfigures}{\renewcommand{\psfigures}{2}}
\newcommand{\namefigures}{\renewcommand{\psfigures}{3}}

\newcommand{\figpart}[1]{{\bf (#1)}\nolinebreak[2]\relax}
\newcommand{\figparts}[2]{{\bf (#1)--(#2)}\nolinebreak[2]\relax}


\macfigures     % STATE

% When calling \figspace, make sure to leave a blank line afterward!!
% \widefigspace is for figures that are more than 28pc wide.
\newlength{\halffigspace} \newlength{\wholefigspace}
\newlength{\figruleheight} \newlength{\figgap}
\newcommand{\setfiglengths}{\ifnum\psfigures=1\setlength{\figruleheight}{\hruleheight}\setlength{\figgap}{1em}\else\setlength{\figruleheight}{0pt}\setlength{\figgap}{0em}\fi}
\newcommand{\figspace}[2]{\ifnum\psfigures=0\leavefigspace{#1}\else%
\setfiglengths%
\setlength{\wholefigspace}{#1}\setlength{\halffigspace}{.5\wholefigspace}%
\rule[-\halffigspace]{\figruleheight}{\wholefigspace}\hspace{\figgap}#2\fi}
\newlength{\widefigspacewidth}
% Make \widefigspace put the figure flush right on the text page.
\newcommand{\widefigspace}[2]{
\ifnum\psfigures=0\leavefigspace{#1}\else%
\setfiglengths%
\setlength{\widefigspacewidth}{28pc}%
\addtolength{\widefigspacewidth}{-\figruleheight}%
\setlength{\wholefigspace}{#1}\setlength{\halffigspace}{.5\wholefigspace}%
\makebox[\widefigspacewidth][r]{#2\hspace{\figgap}}\rule[-\halffigspace]{\figruleheight}{\wholefigspace}\fi}
\newcommand{\leavefigspace}[1]{\setlength{\wholefigspace}{#1}\setlength{\halffigspace}{.5\wholefigspace}\rule[-\halffigspace]{0em}{\wholefigspace}}

% Commands for including figures with macpsfig.
% To use these commands, documentstyle ``macpsfig'' must be specified.
\newlength{\macfigfill}
\makeatother
\newlength{\bbx}
\newlength{\bby}
\newcommand{\macfigure}[5]{\addtocounter{totalart}{1}
\ifnum\psfigures=2%
\setlength{\bbx}{#2}\addtolength{\bbx}{#4}%
\setlength{\bby}{#3}\addtolength{\bby}{#5}%
\begin{flushleft}
\ifdim#4>28pc\setlength{\macfigfill}{#4}\addtolength{\macfigfill}{-28pc}\hspace*{-\macfigfill}\fi%
\mbox{\psfig{figure=./#1.ps,%
bbllx=#2,bblly=#3,bburx=\bbx,bbury=\bby}}
\end{flushleft}%
\else\ifdim#4>28pc\widefigspace{#5}{#1}\else\figspace{#5}{#1}\fi\fi}
\makeatletter

\newlength{\savearraycolsep}
\newcommand{\narrowarray}[1]{\setlength{\savearraycolsep}{\arraycolsep}\setlength{\arraycolsep}{#1\arraycolsep}}
\newcommand{\normalarray}{\setlength{\arraycolsep}{\savearraycolsep}}

\newcommand{\hint}{{\em Hint:\ }}

% Macros from /th/u/clr/mac.tex

\newcommand{\set}[1]{\left\{ #1 \right\}}
\newcommand{\abs}[1]{\left| #1\right|}
\newcommand{\card}[1]{\left| #1\right|}
\newcommand{\floor}[1]{\left\lfloor #1 \right\rfloor}
\newcommand{\ceil}[1]{\left\lceil #1 \right\rceil}
\newcommand{\ang}[1]{\ifmmode{\left\langle #1 \right\rangle}
   \else{$\left\langle${#1}$\right\rangle$}\fi}
        % the \if allows use outside mathmode,
        % but will swallow following space there!
\newcommand{\paren}[1]{\left( #1 \right)}
\newcommand{\bracket}[1]{\left[ #1 \right]}
\newcommand{\prob}[1]{\Pr\left\{ #1 \right\}}
\newcommand{\Var}{\mathop{\rm Var}\nolimits}
\newcommand{\expect}[1]{{\rm E}\left[ #1 \right]}
\newcommand{\expectsq}[1]{{\rm E}^2\left[ #1 \right]}
\newcommand{\variance}[1]{{\rm Var}\left[ #1 \right]}
\renewcommand{\choose}[2]{{{#1}\atopwithdelims(){#2}}}
\def\pmod#1{\allowbreak\mkern12mu({\rm mod}\,\,#1)}
\newcommand{\matx}[2]{\left(\begin{array}{*{#1}{c}}#2\end{array}\right)}
\newcommand{\Adj}{\mathop{\rm Adj}\nolimits}

\newtheorem{theorem}{Theorem}
\newtheorem{lemma}[theorem]{Lemma}
\newtheorem{corollary}[theorem]{Corollary}
\newtheorem{xample}{Example}
\newtheorem{definition}{Definition}
\newenvironment{example}{\begin{xample}\rm}{\end{xample}}
\newcommand{\proof}{\noindent{\em Proof.}\hspace{1em}}
\def\squarebox#1{\hbox to #1{\hfill\vbox to #1{\vfill}}}
\newcommand{\qedbox}{\vbox{\hrule\hbox{\vrule\squarebox{.667em}\vrule}\hrule}}
\newcommand{\qed}{\nopagebreak\mbox{}\hfill\qedbox\smallskip}
\newcommand{\eqnref}[1]{(\protect\ref{#1})}

%%\newcommand{\twodots}{\mathinner{\ldotp\ldotp}}
\newcommand{\transpose}{^{\mbox{\scriptsize \sf T}}}
\newcommand{\amortized}[1]{\widehat{#1}}

\newcommand{\punt}[1]{}

%%% command for putting definitions into boldface
% New style for defined terms, as of 2/23/88, redefined by THC.
\newcommand{\defn}[1]{{\boldmath\textit{\textbf{#1}}}}
\newcommand{\defi}[1]{{\textit{\textbf{#1\/}}}}

\newcommand{\red}{\leq_{\rm P}}
\newcommand{\lang}[1]{%
\ifmmode\mathord{\mathcode`-="702D\rm#1\mathcode`\-="2200}\else{\rm#1}\fi}

%\newcommand{\ckt}[1]{\ifmmode\mathord{\mathcode`-="702D\sc #1\mathcode`\-="2200}\else$\mathord{\mathcode`-="702D\sc #1\mathcode`\-="2200}$\fi}
\newcommand{\ckt}[1]{\ifmmode \sc #1\else$\sc #1$\fi}

%% Margin notes - use \notesfalse to turn off notes.
\setlength{\marginparwidth}{0.6in}
\reversemarginpar
\newif\ifnotes
\notestrue
\newcommand{\longnote}[1]{
  \ifnotes
    {\medskip\noindent Note: \marginpar[\hfill$\Longrightarrow$]
      {$\Longleftarrow$}{#1}\medskip}
  \fi}
\newcommand{\note}[1]{
  \ifnotes
    {\marginpar{\tiny \raggedright{#1}}}
  \fi}


\newcommand{\reals}{\mathbbm{R}}
\newcommand{\integers}{\mathbbm{Z}}
\newcommand{\naturals}{\mathbbm{N}}
\newcommand{\rationals}{\mathbbm{Q}}
\newcommand{\complex}{\mathbbm{C}}

\newcommand{\oldreals}{{\bf R}}
\newcommand{\oldintegers}{{\bf Z}}
\newcommand{\oldnaturals}{{\bf N}}
\newcommand{\oldrationals}{{\bf Q}}
\newcommand{\oldcomplex}{{\bf C}}

\newcommand{\w}{\omega}                 %% for fft chapter

\newenvironment{closeitemize}{\begin{list}
{$\bullet$}
{\setlength{\itemsep}{-0.2\baselineskip}
\setlength{\topsep}{0.2\baselineskip}
\setlength{\parskip}{0pt}}}
{\end{list}}

% These are necessary within a {problems} environment in order to restore
% the default separation between bullets and items.
\newenvironment{normalitemize}{\setlength{\labelsep}{0.5em}\begin{itemize}}
                              {\end{itemize}}
\newenvironment{normalenumerate}{\setlength{\labelsep}{0.5em}\begin{enumerate}}
                                {\end{enumerate}}

%\def\eqref#1{Equation~(\ref{eq:#1})}
%\newcommand{\eqref}[1]{Equation (\ref{eq:#1})}
\newcommand{\eqreftwo}[2]{Equations (\ref{eq:#1}) and~(\ref{eq:#2})}
\newcommand{\ineqref}[1]{Inequality~(\ref{ineq:#1})}
\newcommand{\ineqreftwo}[2]{Inequalities (\ref{ineq:#1}) and~(\ref{ineq:#2})}

\newcommand{\figref}[1]{Figure~\ref{fig:#1}}
\newcommand{\figreftwo}[2]{Figures \ref{fig:#1} and~\ref{fig:#2}}

\newcommand{\liref}[1]{line~\ref{li:#1}}
\newcommand{\Liref}[1]{Line~\ref{li:#1}}
\newcommand{\lirefs}[2]{lines \ref{li:#1}--\ref{li:#2}}
\newcommand{\Lirefs}[2]{Lines \ref{li:#1}--\ref{li:#2}}
\newcommand{\lireftwo}[2]{lines \ref{li:#1} and~\ref{li:#2}}
\newcommand{\lirefthree}[3]{lines \ref{li:#1}, \ref{li:#2}, and~\ref{li:#3}}

\newcommand{\lemlabel}[1]{\label{lem:#1}}
\newcommand{\lemref}[1]{Lemma~\ref{lem:#1}} 

\newcommand{\exref}[1]{Exercise~\ref{ex:#1}}

\newcommand{\handref}[1]{Handout~\ref{#1}}

\newcommand{\defref}[1]{Definition~\ref{def:#1}}

% (1997.8.16: Victor Luchangco)
% Modified \hlabel to only get date and to use handouts counter for number.
%   New \handout and \handoutwithouttitle commands in newmac.tex use this.
%   The date is referenced by <label>-date.
%   (Retained old definition as \hlabelold.)
%   Defined \hforcelabel to use an argument instead of the handouts counter.

\newcounter{handouts}
\setcounter{handouts}{0}

\newcommand{\hlabel}[2]{%
\stepcounter{handouts}
{\edef\next{\write\@auxout{\string\newlabel{#1}{{\arabic{handouts}}{0}}}}\next}
\write\@auxout{\string\newlabel{#1-date}{{#2}{0}}}
}

\newcommand{\hforcelabel}[3]{%          Does not step handouts counter.
\write\@auxout{\string\newlabel{#1}{{#2}{0}}}
\write\@auxout{\string\newlabel{#1-date}{{#3}{0}}}}


% less ugly underscore
% --juang, 2008 oct 05
\renewcommand{\_}{\vrule height 0 pt depth 0.4 pt width 0.5 em \,}


\usepackage{amsmath}
\usepackage{url}
\usepackage{mdwlist}
\usepackage{graphicx}
\usepackage{clrscode3e}
\newcommand{\isnotequal}{\mathrel{\scalebox{0.8}[1]{!}\hspace*{1pt}\scalebox{0.8}[1]{=}}}
\usepackage{listings}
\usepackage{tikz}
\usetikzlibrary{arrows}
\usetikzlibrary{matrix}
\usetikzlibrary{positioning}
\usetikzlibrary{shapes.geometric}
\usetikzlibrary{shapes.misc}
\usetikzlibrary{trees}

\newcommand{\answer}{
 \par\medskip
 \textbf{Answer:}
}

\newcommand{\collaborators}{ \textbf{Collaborators:}
%%% COLLABORATORS START %%%
None.
%%% COLLABORATORS END %%%
}

\newcommand{\answerIa}{ \answer
%%% PROBLEM 1(a) ANSWER START %%%
3
%%% PROBLEM 1(a) ANSWER END %%%
}

\newcommand{\answerIb}{ \answer
%%% PROBLEM 1(b) ANSWER START %%%
3
%%% PROBLEM 1(b) ANSWER END %%%
}

\newcommand{\answerIc}{ \answer
%%% PROBLEM 1(c) ANSWER START %%%
1
%%% PROBLEM 1(c) ANSWER END %%%
}

\newcommand{\answerId}{ \answer
%%% PROBLEM 1(d) ANSWER START %%%
4
%%% PROBLEM 1(d) ANSWER END %%%
}
\newcommand{\answerIe}{ \answer
%%% PROBLEM 1(e) ANSWER START %%%
3
%%% PROBLEM 1(e) ANSWER END %%%
}
\newcommand{\answerIf}{ \answer
%%% PROBLEM 1(f) ANSWER START %%%
2
%%% PROBLEM 1(f) ANSWER END %%%
}
\newcommand{\answerIg}{ \answer
%%% PROBLEM 1(g) ANSWER START %%%
3
%%% PROBLEM 1(g) ANSWER END %%%
}
\newcommand{\answerIh}{ \answer
%%% PROBLEM 1(h) ANSWER START %%%
3
%%% PROBLEM 1(h) ANSWER END %%%
}
\newcommand{\answerIi}{ \answer
%%% PROBLEM 1(i) ANSWER START %%%
4
%%% PROBLEM 1(i) ANSWER END %%%
}
\newcommand{\answerIj}{ \answer
%%% PROBLEM 1(j) ANSWER START %%%
2
%%% PROBLEM 1(j) ANSWER END %%%
}
\newcommand{\answerIk}{ \answer
%%% PROBLEM 1(k) ANSWER START %%%
4
%%% PROBLEM 1(k) ANSWER END %%%
}
\newcommand{\answerIl}{ \answer
%%% PROBLEM 1(l) ANSWER START %%%
8
%%% PROBLEM 1(l) ANSWER END %%%
}
\newcommand{\answerIm}{ \answer
%%% PROBLEM 1(m) ANSWER START %%%
6
%%% PROBLEM 1(m) ANSWER END %%%
}
\newcommand{\answerIn}{ \answer
%%% PROBLEM 1(n) ANSWER START %%%
6
%%% PROBLEM 1(n) ANSWER END %%%
}
\newcommand{\answerIo}{ \answer
%%% PROBLEM 1(o) ANSWER START %%%
7
%%% PROBLEM 1(o) ANSWER END %%%
}
\newcommand{\answerIp}{ \answer
%%% PROBLEM 1(p) ANSWER START %%%
\begin{codebox}
\Procname{$\proc{KTH\_ROOT}(A, K, N)$}
\li $B_0 \gets \proc{ONE}(N)$
\li $i \gets 0$
\li \Repeat \label{li:kthroot-repeat}
\li   $i \gets i + 1$
\li   $\id{B_i} \gets \proc{ADD}(\id{B_{i-1}}, \id{B_{i-1}}, N)}$
\li \Until $\id{B}[N+1] > 0$ or $\proc{POW}(B_i, K) >= A$
\li $R \gets \proc{ZERO}(N)$
\li \For $j \gets i - 1$ \To $0$ \label{li:kth_root-for}
\li   \Do 
\li       $R' \gets \proc{ADD}(R, B_i)$
\li       \If $\proc{POW}(R', K) > A$
\li         \Then 
              $R \gets R'$ \label{kth_root-prog}
            \End
      \End
\li \Return $R$

\Procname{$\proc{POWER}(B, E)$}
\li $P \gets \proc{COPY}(B)$
\li $i \gets 0$
\li \Repeat $i \gets i + 1$
\li   $P \gets \proc{MULTIPLY}(P, B)$
\li \Until $i \gets E$
\li \Return $P$
\end{codebox}
%%% PROBLEM 1(p) ANSWER END %%%
}

\newcommand{\answerIIa}{ \answer
%%% PROBLEM 2(a) ANSWER START %%%
7
%%% PROBLEM 2(a) ANSWER END %%%
}
\newcommand{\answerIIb}{ \answer
%%% PROBLEM 2(b) ANSWER START %%%
3
%%% PROBLEM 2(b) ANSWER END %%%
}
\newcommand{\answerIIc}{ \answer
%%% PROBLEM 2(c) ANSWER START %%%
9
%%% PROBLEM 2(c) ANSWER END %%%
}
\newcommand{\answerIId}{ \answer
%%% PROBLEM 2(d) ANSWER START %%%
1 2 6
%%% PROBLEM 2(d) ANSWER END %%%
}
\newcommand{\answerIIe}{ \answer
%%% PROBLEM 2(e) ANSWER START %%%
1 4 5
%%% PROBLEM 2(e) ANSWER END %%%
}
\newcommand{\answerIIf}{ \answer
%%% PROBLEM 2(f) ANSWER START %%%
2 3
%%% PROBLEM 2(f) ANSWER END %%%
}

\newcommand{\answerIIIa}{ \answer 
%%% PROBLEM 3(a) ANSWER START %%%
fast\_mul
%%% PROBLEM 3(a) ANSWER END %%%
}
\newcommand{\answerIIIb}{ \answer
%%% PROBLEM 3(b) ANSWER START %%%
93496
%%% PROBLEM 3(b) ANSWER END %%%
}
\newcommand{\answerIIIc}{ \answer
%%% PROBLEM 3(c) ANSWER START %%%
4
%%% PROBLEM 3(c) ANSWER END %%%
}
\newcommand{\answerIIId}{ \answer
%%% PROBLEM 3(d) ANSWER START %%%
4
%%% PROBLEM 3(d) ANSWER END %%%
}
\newcommand{\answerIIIe}{ \answer
%%% PROBLEM 3(e) ANSWER START %%%
1 2 3
%%% PROBLEM 3(e) ANSWER END %%%
}
\newcommand{\answerIIIf}{ \answer
%%% PROBLEM 3(f) ANSWER START %%%
1
%%% PROBLEM 3(f) ANSWER END %%%
}

\setlength{\oddsidemargin}{0pt}
\setlength{\evensidemargin}{0pt}
\setlength{\textwidth}{6.5in}
\setlength{\topmargin}{0in}
\setlength{\textheight}{8.5in}

% Fill these in!
\newcommand{\theproblemsetnum}{5}
\newcommand{\releasedate}{October 24, 2011}
\newcommand{\partaduedate}{Monday, October 31}
\newcommand{\tabUnit}{3ex}
\newcommand{\tabT}{\hspace*{\tabUnit}}

\begin{document}

\handout{Problem Set \theproblemsetnum}{\releasedate}

\textbf{Both theory and programming questions} are due {\bf \partaduedate} at
{\bf 11:59PM}.
%
Please download the .zip archive for this problem set, and refer to the
\texttt{README.txt} file for instructions on preparing your solutions.

We will provide the solutions to the problem set 10 hours after the problem set
is due. You will have to read the solutions, and write a brief \textbf{grading
explanation} to help your grader understand your write-up. You will need to
submit the grading explanation by \textbf{Thursday, November 3rd, 11:59PM}. Your
grade will be based on both your solutions and the grading explanation.

\medskip

\hrulefill

\begin{problems}

\problem \points{40} \textbf{The Knight's Shield}

The optimized circuit verifier that you developed on your Amdtel internship was
a huge success and got you on a sure track to landing a sweet offer. You also
got transferred to a research group that is working on the \textit{Knight's
Shield (KS)}\footnote{The code name is Amdtel confidential information. Please
refrain from leaking to TechCrunch.}, a high-stakes project to develop a massive
multi-core chip aimed at the exploding secure cloud computing market.

The KS chip packs $16{,}384$ cores in a die that's the same size as a regular
CPU die. However, each core is very small, and can only do arithmetic operations
using 8-bit or 16-bit unsigned integers (see Table~\ref{table:ks-opcodes}).
Encryption algorithms typically use 2,048-bit integers, so the KS chip will ship
with software that supports arithmetic on large integers. Your job is to help
the KS team assess the efficiency of their software.

\begin{table}[htbp]
\centering
\begin{tabular}{|l|c|c|c|l|}
\hline
Operation & R1 size & R2 size & Result size & Result \\
\hline
\texttt{ZERO} & & & 8 / 16 & 0 (zero) \\ 
\hline
\texttt{ONE} & & & 8 / 16 & 1 (one) \\ 
\hline
\texttt{LSB R1} & 16 & & 8 & R1 \% 256 (least significant byte) \\ 
\hline
\texttt{MSB R1} & 16 & & 8 & R1 / 256 (most significant byte) \\ 
\hline
\texttt{WORD R1} & 8 & & 16 & R1 (expanded to 16-bits) \\ 
\hline
\texttt{ADD R1, R2} & 8 / 16 & 8 / 16 & 16 & \texttt{R1} $+$ \texttt{R2} \\ 
\hline
\texttt{SUB R1, R2} & 8 / 16 & 8 / 16 & 16 & \texttt{R1} $-$ \texttt{R2} $\mod
65536$\\
\hline
\texttt{MUL R1, R2} & 8 & 8 & 16 & \texttt{R1} $\cdot$ \texttt{R2} \\ 
\hline
\texttt{DIV R1, R2} & 16 & 8 & 8 & \texttt{R1} $\div$ \texttt{R2} $\mod 256$\\ 
\hline
\texttt{MOD R1, R2} & 16 & 8 & 8 & \texttt{R1} \% \texttt{R2} \\ 
\hline
\texttt{AND R1, R2} & 8 / 16 & 8 / 16 & 8 / 16 & \texttt{R1} \& \texttt{R2}
(bitwise AND) \\
\hline
\texttt{OR R1, R2} & 8 / 16 & 8 / 16 & 8 / 16 & \texttt{R1} $\|$ \texttt{R2}
(bitwise OR) \\
\hline
\texttt{XOR R1, R2} & 8 / 16 & 8 / 16 & 8 / 16 & \texttt{R1} \verb|^|
\texttt{R2} (bitwise XOR) \\
\hline
\end{tabular}
\caption{Arithmetic operations supported by the KS chip. All sizes are in bits.}
\label{table:ks-opcodes}
\end{table}

The KS library supports arbitrarily large base-256 numbers. The base was chosen
such that each digit is a byte, and two digits make up a 16-bit number. Numbers
are stored as a little-endian sequence of bytes (the first byte of a number is
the least significant digit, for example 65534 = 0xFFFE would be stored as
[0xFE, 0xFF]). For the rest of the problem, assume all the input numbers have
$N$ digits.

Consider the following algorithm for computing $A + B$, assuming both inputs
have $N$ digits.

\begin{codebox}
\Procname{$\proc{Add}(A, B, N)$}
\li $C \gets \proc{Zero}(N + 1)$ \Comment{$\proc{Zero}(k)$ creates a $k$-digit
number, with all digits set to 0s.}
\li $\id{carry} \gets 0$
\li \For $i \gets 1$ \To $N$ \label{li:add-for}
\li   \Do
        $\id{digit} \gets \proc{Word}(A[i]) + \proc{Word}(B[i]) +
        \proc{Word}(\id{carry})$
\li     $\id{C[i]} \gets \proc{Lsb}(\id{digit})$
\li     $\id{carry} \gets \proc{Msb}(\id{digit})$
      \End
\li $C[N + 1] \gets \id{carry}$
\li \Return $C$
\end{codebox}

\begin{problemparts}
\problempart \points{1} What is the running time of \proc{Add}?
\begin{enumerate}
  \item $\Theta(1)$
  \item $\Theta(\log N)$
  \item $\Theta(N)$
  \item $\Theta(N^2)$
  \item $\Theta(N^2 \log N)$
  \item $\Theta(N^{\log_2 3})$
  \item $\Theta(N^{\log_2 6})$
  \item $\Theta(N^3)$
\end{enumerate}
\answerIa

\problempart \points{1} What is the size of \proc{Add}'s output?
\begin{enumerate}
  \item $\Theta(1)$
  \item $\Theta(\log N)$
  \item $\Theta(N)$
  \item $\Theta(N^2)$
  \item $\Theta(N^2 \log N)$
  \item $\Theta(N^{\log_2 3})$
  \item $\Theta(N^{\log_2 6})$
  \item $\Theta(N^3)$
\end{enumerate}
\answerIb

\problempart \points{1} \proc{Add}'s output size suggests an easy lower bound
for the subroutine. Does the running time of \proc{Add} match this lower bound?
\begin{enumerate}
  \item Yes
  \item No
\end{enumerate}
\answerIc

\end{problemparts}

Consider the following brute-force algorithm for computing $A \cdot B$, assuming
both inputs have $N$ digits.

\begin{codebox}
\Procname{$\proc{Multiply}(A, B, N)$}
\li $C \gets \proc{Zero}(2N)$
\li \For $i \gets 1$ \To $N$ \label{li:mul-for1}
\li   \Do
        $\id{carry} \gets 0$
\li     \For $j \gets 1$ \To $N$ \label{li:mul-for2}
\li        \Do
             $\id{digit} \gets A[i] \cdot B[j] + \proc{Word}(C[i + j - 1]) +
                               \proc{Word}(\id{carry})$
\li           $C[i + j - 1] \gets \proc{Lsb}(\id{digit})$
\li           $\id{carry} \gets \proc{Msb}(\id{digit})$
           \End
\li     $\id{C[i + N]} \gets carry$
      \End
\li \Return $C$
\end{codebox}

\begin{problemparts}
\problempart \points{1} What is the running time of \proc{Multiply}?
\begin{enumerate}
  \item $\Theta(1)$
  \item $\Theta(\log N)$
  \item $\Theta(N)$
  \item $\Theta(N^2)$
  \item $\Theta(N^2 \log N)$
  \item $\Theta(N^{\log_2 3})$
  \item $\Theta(N^{\log_2 6})$
  \item $\Theta(N^3)$
\end{enumerate}
\answerId

\problempart \points{1} What is the size of \proc{Multiply}'s output?
\begin{enumerate}
  \item $\Theta(1)$
  \item $\Theta(\log N)$
  \item $\Theta(N)$
  \item $\Theta(N^2)$
  \item $\Theta(N^2 \log N)$
  \item $\Theta(N^{\log_2 3})$
  \item $\Theta(N^{\log_2 6})$
  \item $\Theta(N^3)$
\end{enumerate}
\answerIe

\problempart \points{1} \proc{Multiply}'s output size suggests an easy lower
bound for the subroutine. Does the running time of \proc{Multiply} match this
lower bound?
\begin{enumerate}
  \item Yes
  \item No
\end{enumerate}
\answerIf

\end{problemparts}

Consider the following brute-force algorithm for computing $A \div B$ and $A
\mod B$, assuming both inputs have $N$ digits. The algorithm uses a procedure
$\proc{Copy}(A, N)$ that creates a copy of an $N$-digit number $A$, using
$\Theta(N)$ time.

\begin{codebox}
\Procname{$\proc{DivMod}(A, B, N)$}
\li $Q \gets \proc{Zero}(N)$ \Comment{quotient}
\li $R \gets \proc{Copy}(A, N)$ \Comment{remainder}
\li $S_0 \gets \proc{Copy}(B, N)$ \Comment{$S_i = B \cdot 2^i$}
\li $i \gets 0$
\li \Repeat \label{li:divmod-repeat}
\li   $i \gets i + 1$
\li   $\id{S_i} \gets \proc{Add}(\id{S_{i - 1}}, \id{S_{i - 1}}, N)$
\li \Until $\id{S_i}[N + 1] > 0$ or $\proc{Cmp}(S_i, A, N) \isequal
            \const{Greater}$
\li \For $j \gets i - 1$ \Downto $0$ \label{li:divmod-for}
\li   \Do
        $Q \gets \proc{Add}(Q, Q, N)$
\li     \If $\proc{Cmp}(R, \id{S_j}, N) \isnotequal \const{Smaller}$
\li     \Then
           $R \gets \proc{Subtract}(R, \id{S_j}, N)$
\li        $Q[0] \gets Q[0] \| 1$ \Comment{Faster version of $Q \gets Q + 1$}
        \End
      \End
\li \Return $(Q, R)$
\end{codebox}

\begin{problemparts}
\problempart \points{1} $\proc{Cmp}(A, B, N)$ returns \const{Greater} if $A >
B$, \const{Equal} if $A = B$, and \const{Smaller} if $A < B$, assuming both $A$
and $B$ are $N$-digit numbers. What is the running time for an optimal
$\proc{Cmp}$ implementation?
\begin{enumerate}
  \item $\Theta(1)$
  \item $\Theta(\log N)$
  \item $\Theta(N)$
  \item $\Theta(N^2)$
  \item $\Theta(N^2 \log N)$
  \item $\Theta(N^{\log_2 3})$
  \item $\Theta(N^{\log_2 6})$
  \item $\Theta(N^3)$
\end{enumerate}
\answerIg

\problempart \points{1} $\proc{Subtract}(A, B, N)$ computes $A - B$, assuming
$A$ and $B$ are $N$-digit numbers. What is the running time for an optimal
$\proc{Subtract}$ implementation?
\begin{enumerate}
  \item $\Theta(1)$
  \item $\Theta(\log N)$
  \item $\Theta(N)$
  \item $\Theta(N^2)$
  \item $\Theta(N^2 \log N)$
  \item $\Theta(N^{\log_2 3})$
  \item $\Theta(N^{\log_2 6})$
  \item $\Theta(N^3)$
\end{enumerate}
\answerIh

\problempart \points{1} What is the running time of $\proc{DivMod}$?
\begin{enumerate}
  \item $\Theta(1)$
  \item $\Theta(\log N)$
  \item $\Theta(N)$
  \item $\Theta(N^2)$
  \item $\Theta(N^2 \log N)$
  \item $\Theta(N^{\log_2 3})$
  \item $\Theta(N^{\log_2 6})$
  \item $\Theta(N^3)$
\end{enumerate}
\answerIi

\end{problemparts}

The KS library does not use the $\proc{DivMod}$ implementation above. Instead,
it uses Newton's method to implement $\proc{Div}(A, B, N)$ which computes the
division quotient $A \div B$, assuming both inputs have $N$ digits. $\proc{Div}$
relies on the subroutines defined above. For example, it uses $\proc{Multiply}$
to perform large-number multiplication and $\proc{Add}$ for large-number
addition. $\proc{Mod}(A, B, N)$ is implemented using the identity $A \mod B = A
- (A \div B) \cdot B$.

\begin{problemparts}

\problempart \points{2} How many times does \proc{Div} call $\proc{Multiply}$?
\begin{enumerate}
  \item $\Theta(1)$
  \item $\Theta(\log N)$
  \item $\Theta(N)$
  \item $\Theta(N^2)$
  \item $\Theta(N^2 \log N)$
  \item $\Theta(N^{\log_2 3})$
  \item $\Theta(N^{\log_2 6})$
  \item $\Theta(N^3)$
\end{enumerate}
\answerIj

\problempart \points{2} What is the running time of $\proc{Mod}$?
\begin{enumerate}
  \item $\Theta(1)$
  \item $\Theta(\log N)$
  \item $\Theta(N)$
  \item $\Theta(N^2)$
  \item $\Theta(N^2 \log N)$
  \item $\Theta(N^{\log_2 3})$
  \item $\Theta(N^{\log_2 6})$
  \item $\Theta(N^3)$
\end{enumerate}
\answerIk

\end{problemparts}

Consider the following brute-force algorithm for computing $B^E \mod M$,
assuming all the input numbers have $N$ digits.

\begin{codebox}
\Procname{$\proc{PowMod}(B, E, M, N)$}
\li $R \gets \proc{One}(N)$ \Comment{result}
\li $X \gets \proc{Copy}(B, N)$ \Comment{multiplier}
\li \For $i \gets 1$ \To $N$ \label{li:exp-for1}
\li   \Do
        $\id{mask} \gets 1$
\li     \For $\id{bit} \gets 1$ \To $8$ \label{li:exp-for2}
\li       \Do
            \If $E[i] \textrm{ \& } \id{mask} \isnotequal 0$
\li         \Then
              $R \gets \proc{Mod}(\proc{Multiply}(R, X, N), M, 2N)$
              \label{li:exp-mulmod1}
            \End
\li         $X \gets \proc{Mod}(\proc{Multiply}(X, X, N), M, 2N)$
            \label{li:exp-mulmod2}
\li         $\id{mask} \gets \proc{Lsb}(\id{mask} \cdot 2)$
          \End
      \End
\li \Return $R$
\end{codebox}

\begin{problemparts}

\problempart \points{2} What is the running time for $\proc{PowMod}$?
\begin{enumerate}
  \item $\Theta(1)$
  \item $\Theta(\log N)$
  \item $\Theta(N)$
  \item $\Theta(N^2)$
  \item $\Theta(N^2 \log N)$
  \item $\Theta(N^{\log_2 3})$
  \item $\Theta(N^{\log_2 6})$
  \item $\Theta(N^3)$
\end{enumerate}
\answerIl

\end{problemparts}

Assume the KS library swaps out the brute-force $\proc{Multiply}$ with an
implementation of Karatsuba's algorithm.

\begin{problemparts}
\problempart \points{1} What will the running time for $\proc{Multiply}$ be
after the optimization?
\begin{enumerate}
  \item $\Theta(1)$
  \item $\Theta(\log N)$
  \item $\Theta(N)$
  \item $\Theta(N^2)$
  \item $\Theta(N^2 \log N)$
  \item $\Theta(N^{\log_2 3})$
  \item $\Theta(N^{\log_2 6})$
  \item $\Theta(N^3)$
\end{enumerate}
\answerIm

\problempart \points{2} What will the running time for $\proc{Mod}$ be after
the optimization?
\begin{enumerate}
  \item $\Theta(1)$
  \item $\Theta(\log N)$
  \item $\Theta(N)$
  \item $\Theta(N^2)$
  \item $\Theta(N^2 \log N)$
  \item $\Theta(N^{\log_2 3})$
  \item $\Theta(N^{\log_2 6})$
  \item $\Theta(N^3)$
\end{enumerate}
\answerIn

\problempart \points{2} What will the running time for $\proc{PowMod}$ be after
the optimization?
\begin{enumerate}
  \item $\Theta(1)$
  \item $\Theta(\log N)$
  \item $\Theta(N)$
  \item $\Theta(N^2)$
  \item $\Theta(N^2 \log N)$
  \item $\Theta(N^{\log_2 3})$
  \item $\Theta(N^{\log_2 6})$
  \item $\Theta(N^3)$
\end{enumerate}
\answerIo

\problempart \points{20} Write pseudo-code for $\proc{KthRoot}(A, K, N)$, which
computes $\lfloor \sqrt[K]{A} \rfloor$ using binary search, assuming that $A$
and $K$ are both $N$-digit numbers. The running time for $\proc{KthRoot}(A, K,
N)$ should be $\Theta(N^{2 + \log_2 3})$.

\answerIp

\end{problemparts}

\newpage

\problem \points{18} \textbf{RSA Public-Key Encryption}

The RSA (Rivest-Shamir-Adelman) public-key cryptosystem is a cornerstone of
Internet security. It provides the ``S'' (security) in the HTTPS sessions used
for e-commerce and cloud services that handle private information, such as
e-mail. RSA secures SSH sessions (used to connect to Athena, for example), and
MIT certificates used to log into Stellar. You figure that the KS chip must
perform RSA efficiently, since RSA plays such an important role in cloud
security. This problem will acquaint you with the encryption and decryption
algorithms in RSA.

RSA works as follows. Each user generates two large random primes $p$ and $q$,
and sets his public modulus $m = p \cdot q$. The user then chooses a small
number\footnote{65,537 is a popular choice nowadays} $e$ that is co-prime with
$(p - 1)(q - 1)$, and computes $d = e^{-1} \mod (p - 1)(q - 1)$. The user
announces his public key $(e, m)$ to the world, and keeps $d$ private. In order
to send an encrypted message to our user, another user would encode the message
as a number smaller than $n$, and encrypt it as $c = E(n) = n^e \mod m$. Our
user would decode the message using $D(c) = c^d \mod m$. Assume that keys can be
generated reasonably fast and that $D(E(n)) = n$, for all but a negligible
fraction of values of $n$.

\begin{problemparts}
\problempart \points{1} What is the running time of an implementation of $D(n)$
that uses the KS library in Problem 1, with the optimized version of
\proc{Multiply} (Karatsuba's algorithm), assuming that $n$, $d$ and $m$ are
$N$-byte numbers?
\begin{enumerate}
  \item $\Theta(1)$
  \item $\Theta(\log N)$
  \item $\Theta(N)$
  \item $\Theta(N^2)$
  \item $\Theta(N^2 \log N)$
  \item $\Theta(N^{\log_2 3})$
  \item $\Theta(N^{\log_2 6})$
  \item $\Theta(N^3)$
\end{enumerate}
\answerIIa

\end{problemparts}

You're thinking of using RSA to encrypt important sensitive images, such as
last night's picture of you doing a Keg stand. Formally, a picture has $R \times
C$ pixels ($R$ rows, $C$ columns), and each pixel is represented as 3 bytes that
are RGB color space coordinates\footnote{see
\url{http://en.wikipedia.org/wiki/RGB_color_space}}. The RSA key is $(e, m)$,
where $m$ is an $N$-byte number. An inefficient encryption method would process
each row of pixel data as follows:
\begin{enumerate}
  \item Break the $3C$ bytes of pixel data into groups of $N-1$ bytes
  \item Pad the last group with 0 bytes up to $N-1$ bytes
  \item Encrypt each group of $N-1$ bytes to obtain an $N$-byte output
  \item Concatenate the $N$-byte outputs 
\end{enumerate}

\begin {problemparts}
\problempart \points{1} How many calls to the RSA encryption function $E(n)$
are necessary to encrypt an $R \times C$-pixel image?
\begin{enumerate}
  \item $\Theta(1)$
  \item $\Theta(R C)$
  \item $\Theta(\frac{RC}{N})$
  \item $\Theta(\frac{RN}{C})$
  \item $\Theta(\frac{CN}{R})$
\end{enumerate}
\answerIIb

\problempart \points{1} What is the running time for decrypting an $R \times
C$-pixel image that was encrypted using the method above, using the KS library
in Problem 1, with the optimized version of \proc{Multiply} (Karatsuba's
algorithm)?
\begin{enumerate}
  \item $\Theta(N)$
  \item $\Theta(N^2)$
  \item $\Theta(N^2 \log N)$
  \item $\Theta(N^{\log_2 3})$
  \item $\Theta(N^{\log_2 6})$
  \item $\Theta(R C N)$
  \item $\Theta(R C N^2)$
  \item $\Theta(R C N^2 \log N)$
  \item $\Theta(R C N^{\log_2 3})$
  \item $\Theta(R C N^{\log_2 6})$
  \item $\Theta(R N)$
  \item $\Theta(R N^2)$
  \item $\Theta(R N^2 \log N)$
  \item $\Theta(R N^{\log_2 3})$
  \item $\Theta(R N^{\log_2 6})$
\end{enumerate}
\answerIIc

\problempart \points{5} A fixed point under RSA is a number $n$ such that $E(n)
\equiv n \mod m$, so RSA does not encrypt the number at all. Which of the
following numbers are fixed points under RSA? (True / False)
\begin{enumerate}
  \item $0$
  \item $1$
  \item $2$
  \item $3$
  \item $m - 2$
  \item $m - 1$
\end{enumerate}
\answerIId
  
\problempart \points{5} What other weaknesses does the RSA algorithm have?
(True / False)
\begin{enumerate}
  \item $E(-n) \equiv -E(n) \mod m$
  \item $E(n_1) + E(n_2) \equiv E(n_1 + n_2) \mod m$
  \item $E(n_1) - E(n_2) \equiv E(n_1 - n_2) \mod m$
  \item $E(n_1) \cdot E(n_2) \equiv E(n_1 \cdot n_2) \mod m$
  \item $E(n_1)^{n_2} \equiv E(n_1 ^ {n_2}) \mod m$
\end{enumerate}
\answerIIe

\problempart \points{5} Amdtel plans to use RSA encryption to secretly tell
Gopple when its latest smartphone CPU is ready to ship. Amdtel will send one
message every day to Gopple, using Gopple's public key ($e_G$, $m_G$). The
message will be \const{No} (the number 20079 when using ASCII), until the day
the CPU is ready, then the message will change to \const{Yes} (the number
5858675 when using ASCII). You pointed out to your manager that this security
scheme is broken, because an attacker could look at the encrypted messages, and
know that the CPU is ready when the daily encrypted message changes. This is a
problem of deterministic encryption. If $E(20079)$ always takes the same value,
an attacker can distinguish $E(20079)$ from $E(5858675)$. How can the problem of
deterministic encryption be fixed? (True / False)

\begin{enumerate}
  \item Append the same long number (the equivalent of a string such as
  'XXXPADDINGXXX') to each message, so the messages are bigger.
  \item Append a random number to each message. All random numbers will have the
  same size, so the receiver can recognize and discard them.
  \item Use a different encryption key to encrypt each message, and use Gopple's
  public exponent and modulus to encrypt the decryption key for each message.
  \item Use an uncommon encoding, such as UTF-7, so that the attacker will not
  know the contents of the original messages.
  \item Share a ``secret'' key with Gopple, so that the attacker can't use the
  knowledge on Gopple's public exponent and modulus.
\end{enumerate}
\answerIIf

\end{problemparts}

\newpage

\problem \points{42} \textbf{Image Decryption}

Your manager wants to show off the power of the Knight's Shield chip by
decrypting a live video stream directly using the RSA public-key crypto-system.
RSA is quite resource-intensive, so most systems only use it to encrypt the key
of a faster algorithm. Decrypting live video would be an impressive technical
feat!

Unfortunately, the performance of the KS chip on RSA decryption doesn't come
even close to what's needed for streaming video. The hardware engineers said the
chip definitely has enough computing power, and blamed the problem on the
RSA implementation. Your new manager has heard about your algorithmic chops, and
has high hopes that you'll get the project back on track. The software engineers
suggested that you benchmark the software using images because, after all, video
is just a sequence of frames.

The code is in the \texttt{rsa} directory in the zip file for this problem set.

\begin{problemparts}
\problempart \points{2} Run the code under the python profiler with the command
below, and identify the method inside \texttt{bignum.py} that is most suitable
for optimization. Look at the methods that take up the most CPU time, and
choose the first method whose running time isn't proportional to the size of its
output.

\texttt{python -m cProfile -s time rsa.py < tests/1verdict\_32.in}

\textit{Warning:} the command above can take 1-10 minutes to complete, and
bring the CPU usage to 100\% on one of your cores. Plan accordingly. If
you have installed PyPy successfully, you should replace \texttt{python} with
\texttt{pypy} in the command above for a 2-10x speed improvement. 

What is the name of the method with the highest CPU usage?
\answerIIIa

\problempart \points{1} How many times is the method called?
\answerIIIb

\problempart \points{1} The troublesome method is implementing a familiar
arithmetic operation. What is the tightest asymptotic bound for the worst-case
running time of the method that contains the bottleneck? Express your answer in
terms of $N$, the number of digits in the input numbers.
\begin{enumerate}
  \item $\Theta(N)$.
  \item $\Theta(N \log n)$
  \item $\Theta(N \log^2 n)$
  \item $\Theta(N^{\log_{2} 3})$
  \item $\Theta(N^2)$
  \item $\Theta(N^{\log_{2} 7})$
  \item $\Theta(N^3)$
\end{enumerate}
\answerIIIc

\problempart \points{1} What is the tightest asymptotic bound for the worst-case
running time of division? Express your answer in terms of $N$, the number of
digits in the input numbers.
\begin{enumerate}
  \item $\Theta(N)$.
  \item $\Theta(N \log n)$
  \item $\Theta(N \log^2 n)$
  \item $\Theta(N^{\log_{2} 3})$
  \item $\Theta(N^2)$
  \item $\Theta(N^{\log_{2} 7})$
  \item $\Theta(N^3)$
\end{enumerate}
\answerIIId

\end{problemparts}

We have implemented a visualizer for your image decryption output, to help you
debug your code. The visualizer will also come in handy for answering the
question below. To use the visualizer, first produce a trace.

\texttt{TRACE=jsonp python rsa.py < tests/1verdict\_32.in > trace.jsonp}

On Windows, use the following command instead.

\texttt{rsa\_jsonp.bat < tests/1verdict\_32.in > trace.jsonp}

Then use Google Chrome to open
\texttt{visualizer/bin/visualizer.html}

\begin{problemparts}
\problempart \points{6} The test cases that we supply highlight the problems of
RSA that we discussed above. Which of the following is true? (True / False)
\begin{enumerate}
  \item Test \texttt{1verdict\_32} shows that RSA has fixed points.
  \item Test \texttt{1verdict\_32} shows that RSA is deterministic.
  \item Test \texttt{2logo\_32} shows that RSA has fixed points.
  \item Test \texttt{2logo\_32} shows that RSA is deterministic.
  \item Test \texttt{5future\_1024} shows that RSA has fixed points.
  \item Test \texttt{5future\_1024} shows that RSA is deterministic.
\end{enumerate}
\answerIIIe

\problempart \points{1} Read the code in \texttt{rsa.py}. Given a decrypted
image of $R \times C$ pixels ($R$ rows, $C$ columns), where all the pixels
are white (all the image data bytes are 255), how many times will
\texttt{powmod} be called during the decryption operation in
\texttt{decrypt\_image}?
\begin{enumerate}
  \item $\Theta(1)$
  \item $\Theta(R C)$
  \item $\Theta(\frac{RC}{N})$
  \item $\Theta(\frac{RN}{C})$
  \item $\Theta(\frac{CN}{R})$
\end{enumerate}
\answerIIIf

\problempart \points{30} The multiplication and division operations in
\texttt{big\_num.py} are implemented using asymptotically efficient algorithms
that we have discussed in class. However, the sizes of the numbers involved in
RSA for typical key sizes aren't suitable for complex algorithms with high
constant factors. Add new methods to \texttt{BigNum} implementing multiplication
and division using straight-forward algorithms with low constant factors, and
modify the main multiplication and division methods to use the simple algorithms
if at least one of the inputs has 64 digits (bytes) or less. Please note that
you are not allowed to import any additional Python libraries and our test will
check this.
\end{problemparts}

The KS software testing team has put together a few tests to help you check your
code's correctness and speed. \texttt{big\_num\_test.py} contains unit tests
with small inputs for all \texttt{BigNum} public methods.
\texttt{rsa\_test.py} runs the image decryption code on the test cases in the
\texttt{tests/} directory.

You can use the following command to run all the image decryption tests.

\texttt{python rsa\_test.py}

To work on a single test case, run the simulator on the test case with the
following command.

\texttt{python rsa.py < tests/1verdict\_32.in > out}

Then compare your output with the correct output for the test case.

\texttt{diff out tests/1verdict\_32.gold}

For Windows, use \texttt{fc} to compare files.

\texttt{fc out tests/1verdict\_32.gold}

While debugging your code, you should open a new Terminal window (Command Prompt
in Windows), and set the \texttt{KS\_DEBUG} environment variable (\texttt{export
KS\_DEBUG=true}; on Windows, use \texttt{set KS\_DEBUG=true}) to use a slower
version of our code that has more consistency checks.

When your code passes all tests, and runs reasonably fast (the tests should
complete in less than 90 seconds on any reasonably recent computer using PyPy,
or less than 600 seconds when using CPython), upload your modified
\texttt{big\_num.py} to the course submission site. Our automated grading code
will use our versions of \texttt{test\_rsa.py}, \texttt{rsa.py} and
\texttt{ks\_primitives.py} / \texttt{ks\_primitives\_unchecked.py}, so please do
not modify these files.

\end{problems}
\end{document}
